%--------------------------------------------------------------------------------------
% Este arquivo contém a sua introdução, objetivos e organização do trabalho
%--------------------------------------------------------------------------------------
\chapter{Introdução}

As técnicas de avaliação de desempenhos de sistemas computacionais são úteis para que, através da estimativa do desempenho de um sistema similar ao que deseja-se desenvolver, possa-se determinar as configurações ideais dos componentes pertencentes ao sistema. A técnica QSN-RI fornece resultados objetivos quanto à utilização dos componentes, seu tempos de espera e de resposta e tamanho da fila. Ela recebe como parâmetros o tempo de serviço dos componentes, número de tarefas, o intervalo entre visitas, o número de repetições, a quantidade total de visita aos componentes e a ordem das visitas à eles.

Quando aliada à experimentos de $2^k$ fatores, é possível obter um diagnóstico completo quanto aos equipamentos que mais impactam no desempenho do sistema como um todo. Como este experimento é realizado para cada combinação de configuração dos componentes, é possível determinar os modelos mais adequados dos equipamentos.

Dessa forma, será conduzido um experimento com $k$ = 5 fatores, resultando em 32 combinações. A partir dos resultados obtidos, serão determinados os impactos de cada componente no sistema, assim como a configuração mais adequada para um melhor desempenho dele.

% \lipsum[5-10]


% \section{Objetivos gerais}

% \lipsum[1]

% \section{Objetivos específicos}

% \begin{itemize}
% 	\item blablablabla;
%     \item blablablabla;
%     \item blablablabla.
% \end{itemize}

% \section{Organização do trabalho}

% \lipsum[10-12]
